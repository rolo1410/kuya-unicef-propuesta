\section{Perfil de la Institución}
\label{sec:perfil_institucion}

\subsection{Información General}

KUYACODE S.A.S, es una empresa Ecuatoriana generadora de soluciones tecnológicas innovadoras, fundada en 2025, conformada por un equipo multidisciplinario con más de 10 años de experiencia en el desarrollo de software y consultoría tecnológica. Nuestra misión es proporcionar soluciones que impulsen la eficiencia y la innovación en diversas industrias, incluyendo el sector humanitario.

\subsection{Experiencia Relevante}

Como consultores independientes a lo largo de nuestra trayectoria, hemos trabajado con múltiples organizaciones internacionales, incluyendo agencias de la ONU, ONGs y entidades gubernamentales. como empresa KUYACODE S.A.S, hemos participado en 2 proyecto elevantes.

Algunos de nuestros proyectos más destacados incluyen:
\begin{itemize}
    \item \textbf{2021 Implementación de la plataforma DHIS2}\footnote{ District Health Information System, Sistema de información de salud distrital}
    
        En el 2021 y parte del 2022 bajo el auspicio de la OPS/OMS, participamos en la implementación de la plataforma DHIS2 (District Health Information System) para el Ministerio de Salud Pública de Ecuador, con el objetivo de mejorar la gestión y análisis de datos de salud a nivel nacional.

        La plataforma DHIS2 fue implementada para recopilar los datos nominales de vacunación COVID-19 y ESAVIS en todo el país a nivel de establecimiento, permitiendo un seguimiento más eficiente de las campañas de vacunación y la gestión de eventos adversos relacionados con la inmunización.

        La configuración de la plataforma incluyó la integración con los servicios del Registro Civil, para la agilización y registro automático de datos de los pacientes vacunados, cono nombres, apellidos, cédula de identidad, fecha de nacimiento, sexo, estado civil.

        Incluyo tambien la integración con la plataforma de distribución de establecimientos de salud del Ministerio de Salud Pública, para la gestión y actualización automática de la información de los establecimientos de salud a nivel nacional.

        Nuestra labor incluyó la configuración del sistema, capacitación del personal y soporte técnico continuo para asegurar una adopción efectiva de la plataforma.

    \item \textbf{2022, Sistema de información Regional ESAVIS}\footnote{Evento Supuestamente Atribuible a la Vacunación o Inmunización} 

        Con el auspicio de la OPS/OMS REGIONAL, brindamos consultorías especializadas para la implementación de sistemas de información de ESAVIS REGIONAL en Paraguay, Ecuador y El Salvador.

        Las consulorias incluyeron el análisis de requerimientos, diseño e implementación de la plataforma tecnológicas para el registro de los eventos supuestamente atribuibles a la vacunación o inmunización (ESAVIS) ajustados a la realidad tecnológica de cada país.

        Las implementaciones brindaron un mecanismo para la notificación regional de ESAVIS, permitiendo el analisis y monitoreo de los eventos relacionados con la vacunación de una forma estandarizada y eficiente.
        
        Los sistemas implementados permiten la integración y estandarización de datos, de las deferentes fuentes, mejorando la eficiencia y efectividad de los programas de vacunación en los países beneficiarios.
    
    \item \textbf{2024 - Actualidad, YAKU, IUSTITIA PERMANENS}.

        KUYACODE S.A.S ha brindado servicios de consultoría al consorcio de abogados IUSTITIA PERMANENS, Abg Asociados para el desarrollo de la plataforma multitenant de registro y monitoreo de consumo de agua potable comunitaria YAKU. El proyecto YAKU constituye una solución tecnológica integral que permite el registro, monitoreo y análisis del consumo de agua potable en comunidades rurales, facilitando la toma de decisiones informadas para mejorar la gestión del recurso hídrico.

        El sistema incluye una aplicación móvil para la recolección de datos en campo en modo desconectado, es utilizando para el registro de de las lecturas de consumo de agua en los medidores. 

        La plataforma gestiona el registro de clientes, acometidas, medidores, predios, consumos de agua, generación de facturas y reportes contables, proporcionando una herramienta completa para la administración de servicios de agua potable en entornos comunitarios.

        La plataforma incluye una aplicación movil para la recolección de datos en campo en modo desconectado, es utilizando para el registro de de las lecturas de consumo de agua en los medidores por parte de los operadores, facilitando la labor de recolección de datos en entornos rurales.

        Actualmente, la plataforma YAKU se encuentra implementada para la comunidad de AINCHE, ubicada en la provincia de Chimborazo, Ecuador, y está en proceso de expansión a otras comunidades.

        URL: \url{https://yaku.kuyacode.com}

    \item \textbf{2024-2025, Plataforma de Monitoreo de Flota y Logística}

        KUYACODE S.A.S ha brindado servicios de consultoría a la empresa QUALITYFAST S.A. en el desarrollo y mejora de su plataforma de monitoreo en tiempo real de vehículos y logística en transporte. Esta plataforma permite a QUALITYFAST S.A. rastrear y gestionar eficientemente su flota de vehículos, optimizando las operaciones logísticas y mejorando la eficiencia del transporte.

        La plataforma incluye funcionalidades avanzadas de seguimiento GPS, gestión de rutas, monitoreo en tiempo real de vehículos, gestión de alertas y notificaciones, gestiones de flotas, contratos, clientes, responsables, dispositivos, entre otros.
        
        Actualmente QUALITYFAST S.A está migrando su anterior plataforma a esta nueva versión que implementa una nueva arquitectura basada en microservicios para mejorar su escalabilidad y rendimiento, ademas de integrar nuevas funcionalidades, mejoras funcionales y de seguridad.

        URL: \url{http://rastreoweb.net/}, \url{http://devtenant1.rastreoweb.com/}
\end{itemize}

\subsection{Referencias}
Podemos proporcionar referencias de nuestros clientes.
\begin{itemize}
    \item \textbf{IUSTITIA PERMANENS, Abg Asociados:} Proyecto YAKU
    \begin{itemize}
        \item Abg. Ricardo Parra Chavez
        \item Teléfono: +593 98 764 9183
        \item Correo Electrónico: ricardo.parra@gad.com
    \end{itemize}
    \item \textbf{Ing. Francisco Quiroga, CEO} Proyecto QUALITYFAST S.A., Rastreo Vehicular y Logística
    \begin{itemize}
        \item Ing. Francisco Quiroga
        \item Teléfono: +593 99 929 3029
        \item Correo Electrónico: fastq2000@gmail.com
    \end{itemize}
\end{itemize}

\subsection{Certificaciones}
Contamos con diversas certificaciones que respaldan nuestra experiencia y compromiso con la calidad, incluyendo:
\begin{itemize}
    \item Certificación en Gestión de Proyectos (PMP)
    \item Certificación en Desarrollo Ágil (Scrum Master)
    \item Certificación en Análisis de Datos (Google Data Analytics)
\end{itemize}
\subsection{Perfiles del Equipo}
Nuestro equipo está compuesto por profesionales altamente calificados en diversas áreas, incluyendo desarrollo de software, gestión de proyectos, análisis de datos y consultoría tecnológica. A continuación, se listan los perfiles considerados para el desarrollo de este proyecto:
\begin{itemize}
        \item \textbf{Ing. Rolando Casigña Parra} - CEO, Arquitecto de Soluciones: Con 13 años de experiencia en desarrollo de software y gestión de proyectos tecnológicos. Actualmente involucrado en soluciones de análisis de datos y trabajo de datos. con experiencia en tecnologías JAVA, nodejs, python, bases de datos relacionales.
        \item \textbf{Msg. Eco. Nataly Verdugo Morales} - PM: Economista con 15 años de experiencia en la gestión de proyectos de análisis económico, trabajo con recursos humanos y gestión logistica.
        \item \textbf{Msg. Ing. Nelson López Naranjo} - Lider Técnico: Ingeniero en sistemas con más de 15 años de experiencia en desarrollo de software y gestión de equipos técnicos, especialista de arquitecturas de software y metodologias ágiles, con experiencia y manejo de tecnologías JAVA, nodejs, bases de datos relacionales.
        \item \textbf{Msg. Ing. Roberto Maldonado Palacios} - Desarrollador Full Stack: Ingeniero en sistemas con 15 años de experiencia en desarrollo de software, especializado en aplicaciones web y móviles, experiencia con tecnologías front-end y back-end con lenguajes JAVA, nodejs, angular, react.
        \item \textbf{Msg. Ing. Ximena Celi Celi} - Analista de datos: Ingeniera en sistemas con experiencia en análisis de datos y visualización, especializada en herramientas opensource Python, streamlet, jasperreports.
        \item  \textbf{Tec. Feddy Parra Aguiar} - Desarrollador Móvil: Técnico en desarrollo de software con experiencia en aplicaciones móviles hibrídas utilizando react native.
        \item  \textbf{Ing. Ivete Castrillon Zamora} - Especialista en Aseguramiento de Calidad: Ingeniera en sistemas con experiencia en pruebas de software, automatización de pruebas y aseguramiento de calidad, con conocimientos en herramientas como Selenium, JUnit, k6.
\end{itemize}
Las hojas de vida detalladas de cada miembro del equipo están disponibles en los anexos de este documento.