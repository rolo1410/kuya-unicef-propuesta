\section{Metodología de Desarrollo}

\subsection{Enfoque Metodológico}

Para el desarrollo de este proyecto se adoptará una metodología ágil basada en \textbf{Scrum}, que permitirá una entrega iterativa e incremental de los productos, garantizando la adaptabilidad a los cambios y la participación activa del cliente durante todo el proceso de desarrollo.

\subsection{Fases del Proyecto}

Tomando en cuenta las fases planteadas en los terminos de referencia, se ajustará la metodología para cumplir con los entregables y cronogramas establecidos.

\begin{itemize}
    \item \textbf{Producto 1}: Documento de diagnóstico, restructuración y mejora del sistema de alerta temprana SAT (Plan de trabajo).

    Principales actores:
    \begin{table}[h]
        \centering
        \begin{tabular}{|p{3cm}|p{5cm}|p{7cm}|}
            \hline
            \textbf{Rol} & \textbf{Entidad} & \textbf{Descripción} \\
            \hline
            Coordinador Técnico & Ministerio de Educación & Enlace técnico y validador de los entregables por parte del cliente \\
            \hline
            Arquitecto de Software & KUYACODE & Encargado del diseño de la arquitectura técnica y tecnológica del sistema \\
            \hline
            Administrador de Proyectos (PM) & KUYACODE & Responsable de la gestión, planificación y seguimiento del proyecto \\
            \hline
        \end{tabular}
        \caption{Principales actores involucrados en el Producto 1}
        \label{tab:actores_producto1}
    \end{table}
    
    \item \textbf{Producto 2}: Documento para el rediseño del Sistema de Alerta Temprana SAT conforme a los lineamientos del Ministerio de Educación, Deporte y Cultura.\\
    \begin{table}[htbp]
        \centering
        \begin{tabular}{|p{3cm}|p{5cm}|p{7cm}|}
            \hline
            \textbf{Rol} & \textbf{Entidad} & \textbf{Descripción} \\
            \hline
            Coordinador Técnico & Ministerio de Educación & Enlace técnico y validador de los entregables por parte del cliente \\
            \hline
            Arquitecto de Software & KUYACODE & Responsable de recopilar y documentar los requerimientos funcionales y no funcionales del sistema \\
            \hline
            Administrador de proyecto& KUYACODE & Responsable de la gestión y comunicación del proyecto e involucrados \\
            \hline
        \end{tabular}
        \caption{Principales actores involucrados en el Producto 2}
        \label{tab:actores_producto2}
    \end{table}
    \\

    \item \textbf{Producto 3}: Correcciones, reestructuración y mejora de la plataforma de acuerdo con el detalle de Requerimiento Funcional sobre la plataforma de Desarrollo proporcionada por el Ministerio de Educación, Deporte y Cultura.\\

    \begin{table}[h]
        \centering
        \begin{tabular}{|p{3cm}|p{5cm}|p{7cm}|}
            \hline
            \textbf{Rol} & \textbf{Entidad} & \textbf{Descripción} \\
            \hline
            Coordinador Técnico & Ministerio de Educación & Enlace técnico y validador de los entregables por parte del cliente \\
            \hline
            Administrador de proyecto & KUYACODE & Responsable de la gestión y comunicación del proyecto e involucrados \\
            \hline
            Arquitecto de Software & KUYACODE & Responsable de la definición de la arquitectura técnica y tecnológica del sistema, desarrollo de prototipos y validación de soluciones \\
            \hline
            Desarrollador Senior & KUYACODE & Encargado de la implementación de las correcciones y mejoras funcionales definidas en los terminos de referencia y levantamiento de requerimientos \\
            \hline
            Desarrollador Senior WEB & KUYACODE & Encargado de la implementación de las correcciones y mejoras funcionales web definidas en los terminos de referencia y levantamiento de requerimientos \\
            \hline
            Analista de datos & KUYACODE & Encargado de la implementación de la analítica de datos del sistema \\
            \hline
        \end{tabular}
        \caption{Principales actores involucrados en el Producto 3}
        \label{tab:actores_producto3}
        \end{table}

    \item \textbf{Producto 4}: Informe de pruebas del funcionamiento del Sistema de Alerta Temprana SAT.\\
    
    \begin{table}[htbp]
        \centering
        \begin{tabular}{|p{3cm}|p{5cm}|p{7cm}|}
            \hline
            \textbf{Rol} & \textbf{Entidad} & \textbf{Descripción} \\
            \hline
            Coordinador Técnico & Ministerio de Educación & Validador de los resultados de pruebas por parte del cliente \\
            \hline
            Administrador de proyecto & KUYACODE & Responsable de la gestión y comunicación del proyecto e involucrados \\
            \hline
            QA Engineer & KUYACODE & Responsable de diseñar y ejecutar las pruebas funcionales y no funcionales del sistema \\
            \hline
            Arquitecto de Software & KUYACODE & Supervisor técnico de las pruebas y validación de resultados \\
            \hline
        \end{tabular}
        \caption{Principales actores involucrados en el Producto 4}
        \label{tab:actores_producto4}
    \end{table}

    \item \textbf{Producto 5}: Informe de implementación definitiva del sistema en un ambiente productivo.\\
    
    \begin{table}[h]            
        \centering
        \begin{tabular}{|p{3cm}|p{5cm}|p{7cm}|}
            \hline
            \textbf{Rol} & \textbf{Entidad} & \textbf{Descripción} \\
            \hline
            Administrador de Sistemas & Ministerio de Educación & Responsable de la infraestructura y soporte del ambiente productivo \\
            \hline
            Coordinador Técnico & KUYACODE & Responsable del despliegue y configuración en ambiente productivo \\
            \hline
            Administrador de Proyectos (PM) & KUYACODE & Coordinador de la implementación y seguimiento de actividades \\
            \hline
        \end{tabular}   
        \caption{Principales actores involucrados en el Producto 5}
        \label{tab:actores_producto5}
    \end{table}

    \item \textbf{Producto 6}: Plan de capacitación que incluya fechas de las sesiones de capacitación acordadas, aprobado por la Dirección de Gestión de Riesgos.
    
    \begin{table}[h]
        \centering
        \begin{tabular}{|p{3cm}|p{5cm}|p{7cm}|}
            \hline
            \textbf{Rol} & \textbf{Entidad} & \textbf{Descripción} \\
            \hline
            Capacitador Técnico & KUYACODE & Responsable del diseño e impartición de las capacitaciones técnicas \\
            \hline
            Administrador de Proyectos (PM) & KUYACODE & Impartir las capacitaciones funcionales y supervisar su ejecución \\
            \hline
            Arquitecto de software & KUYACODE & Impartir las capacitaciones técnicas y supervisar su ejecución \\
            \hline
        \end{tabular}
        \caption{Principales actores involucrados en el Producto 6}
        \label{tab:actores_producto6}
    \end{table} 
    \item \textbf{Producto 7}: Informe de soporte técnico que incluya los ajustes realizados. Acta de cierre técnico aprobada por la Dirección de Gestión de Riesgos y la Coordinación Nacional de TICS.

    \begin{table}[h]
        \centering
        \begin{tabular}{|p{3cm}|p{5cm}|p{7cm}|}
            \hline
            \textbf{Rol} & \textbf{Entidad} & \textbf{Descripción} \\
            \hline
            Soporte Técnico & KUYACODE & Responsable de brindar soporte técnico y realizar ajustes necesarios \\
            \hline
            Administrador de Proyectos (PM) & KUYACODE & Coordinador del cierre técnico y documentación final \\
            \hline
            Director de Gestión de Riesgos & Ministerio de Educación & Aprobador del acta de cierre técnico \\
            \hline
            Coordinador Nacional de TICS & Ministerio de Educación & Co-aprobador del acta de cierre técnico \\
            \hline
        \end{tabular}
        \caption{Principales actores involucrados en el Producto 7}
        \label{tab:actores_producto7}
    \end{table}
    \item \textbf{Producto 8}: Certificado de garantía de funcionamiento.
    \begin{table}[ht]
        \centering
        \begin{tabular}{|p{3cm}|p{5cm}|p{7cm}|}
            \hline
            \textbf{Rol} & \textbf{Entidad} & \textbf{Descripción} \\
            \hline
            Gerente de Proyecto & KUYACODE & Responsable de emitir el certificado de garantía y comprometerse con el soporte \\
            \hline
            Arquitecto de Software & KUYACODE & Validador técnico de la calidad y funcionamiento del sistema \\
            \hline
            Director de Gestión de Riesgos & Ministerio de Educación & Receptor y validador del certificado de garantía \\
            \hline
        \end{tabular}
        \caption{Principales actores involucrados en el Producto 8}
        \label{tab:actores_producto8}
    \end{table}
\end{itemize}
\subsection{Herramientas y Tecnologías}
\subsubsection{Gestión de Proyecto}
La propuesta incluye la intervención de un Administrador de proyectos (PM) que utilizará herramientas ágiles para la gestión y seguimiento del proyecto, tales como:
\begin{itemize}
    \item \textbf{Libre Project}: Gestión de proyectos, para la planificación y seguimiento de tareas, hitos y recursos del proyecto.
    \item \textbf{Confluence}: Plataforma de colaboración para la documentación del proyecto, permitiendo la creación y gestión de documentos técnicos, actas de reuniones y otros recursos relevantes.
    \item \textbf{skalydraw}: Herramienta para la creación de diagramas y flujogramas que faciliten la visualización de procesos y arquitecturas del sistema.
\end{itemize}
\subsubsection{Control de Versiones}
El manejo del código fuente se realizará utilizando las siguientes herramientas:
\begin{itemize}
    \item \textbf{Git}: Plataforma de control de versiones gestionado por el ministerio Ministerio de Educación, Cultura y deporte, con acceso para los desarrolladores y el equipo de QA.
    Si el Ministerio de Educación lo requiere, se creará una organización en GitHub con acceso privado a un equipo limitado.
\end{itemize}
\subsubsection{Documentación}
La documentación generada del proyecto será almacenada y gestionada utilizando:
\begin{itemize}
    \item \textbf{Git-Wiki}: Permite generar una wiki con la documentación del proyecto, asi tambien el versionamiento del mismo.
    \item \textbf{Swagger}: Documentación de APIs RESTful, permitiendo una fácil comprensión y uso por parte de los desarrolladores y clientes de las aplicaciones.
    \item \textbf{Docs}: Generador de documentación técnica a partir de comentarios en el código fuente.
\end{itemize}
\subsubsection{Desarrollo y Testing}
El desarrollo y testing se realizará utilizando las siguientes herramientas y tecnologías:
\begin{itemize}
    \item \textbf{IDE}: Visual Studio Code, IntelliJ IDEA, Eclipse IDEA
    \item \textbf{Lenguajes de Programación}: Java, JavaScript/TypeScript, Python
    \item \textbf{Frameworks}: Spring Boot, React, Angular, Node.js
    \item \textbf{Testing}: Jest, JUnit, Selenium
\end{itemize}
\subsection{Aseguramiento de Calidad}
La validación y aseguramiento de la calidad del software se llevará a cabo mediante las siguientes prácticas:
\begin{itemize}
    \item \textbf{Codestyler}: Herramienta para el análisis y formateo del código fuente, asegurando el cumplimiento de las convenciones de estilo.
    \item \textbf{SonarQube}: Análisis estático de código para identificar vulnerabilidades, bugs y code smells.
    \item \textbf{Pruebas Unitarias}: Implementación de pruebas unitarias para validar el correcto funcionamiento de los componentes individuales del software integradas en la construcción de la solución.
    \item \textbf{Pruebas de integración}: Validación de la interacción entre diferentes módulos del sistema.
    \item \textbf{Pruebas de aceptación}: Validación final por parte del cliente para asegurar que el producto cumple con los requerimientos especificados mediante la revisión y aprobación funcional.
\end{itemize}
\subsubsection{Proceso de Testing}
La estrategia de testing incluirá:
\begin{itemize}
    \item \textbf{Pruebas automatizadas}: Unitarias, integración y funcionales
    \item \textbf{Pruebas manuales}: Exploratorias y de usabilidad
    \item \textbf{Performance testing}: Pruebas de carga y rendimiento
\end{itemize}
\subsection{Comunicación y Seguimiento}
\subsubsection{Reuniones Regulares}
La propuesta incluye la realización de reuniones regulares para el seguimiento del proyecto, incluyendo:
\begin{itemize}
    \item \textbf{Daily Standup}: Reuniones diarias de 15 minutos, con el equipo de desarrollo para revisar el progreso y obstáculos, con la participación del PM, desarrolladores y líderes técnicos y contraparte funcional.
    \item \textbf{Sprint Planning}: Planificación al inicio de cada sprint, reunión para definir las tareas y objetivos del sprint cada 15 días. con los participación de el PM, desarrolladores, líderes técnicos y contraparte funcional.
    \item \textbf{Sprint Review}: Demostración al final de cada sprint cada 5 días.
\end{itemize}
\subsubsection{Reportes y Métricas}
\begin{itemize}
    \item Burndown charts para seguimiento del progreso
    \item Reportes semanales de avance
    \item Métricas de calidad y rendimiento
    \item Documentación de lecciones aprendidas
\end{itemize}
\subsection{Gestión de Riesgos}
\begin{itemize}
    \item Identificación temprana de riesgos técnicos y de proyecto
    \item Plan de contingencia para escenarios críticos
    \item Backup y recuperación de datos
    \item Documentación de decisiones técnicas importantes
\end{itemize}