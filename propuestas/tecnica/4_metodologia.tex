\section{Metodología de Desarrollo}

\subsection{Enfoque Metodológico}

Para el desarrollo de este proyecto se adoptará una metodología ágil basada en \textbf{Scrum}, que permitirá una entrega iterativa e incremental del producto, garantizando la adaptabilidad a los cambios y la participación activa del cliente durante todo el proceso de desarrollo.

\subsection{Fases del Proyecto}

\subsubsection{Fase 1: Análisis y Planificación}
\begin{itemize}
    \item Análisis detallado de requerimientos
    \item Definición de la arquitectura del sistema
    \item Planificación de sprints y cronograma
    \item Configuración del entorno de desarrollo
\end{itemize}

\subsubsection{Fase 2: Desarrollo Iterativo}
\begin{itemize}
    \item Sprints de 2-3 semanas
    \item Desarrollo de funcionalidades por prioridad
    \item Revisiones y demos regulares con el cliente
    \item Pruebas continuas e integración
\end{itemize}

\subsubsection{Fase 3: Pruebas y Validación}
\begin{itemize}
    \item Pruebas unitarias y de integración
    \item Pruebas de usuario y aceptación
    \item Corrección de errores y optimización
    \item Validación de requerimientos
\end{itemize}

\subsubsection{Fase 4: Despliegue y Entrega}
\begin{itemize}
    \item Preparación del entorno de producción
    \item Migración de datos (si aplica)
    \item Capacitación de usuarios
    \item Documentación técnica y de usuario
\end{itemize}

\subsection{Herramientas y Tecnologías}

\subsubsection{Gestión de Proyecto}
\begin{itemize}
    \item \textbf{Jira/Trello}: Gestión de tareas y seguimiento
    \item \textbf{Confluence}: Documentación colaborativa
    \item \textbf{Slack/Teams}: Comunicación del equipo
\end{itemize}

\subsubsection{Control de Versiones}
\begin{itemize}
    \item \textbf{Git}: Control de versiones distribuido
    \item \textbf{GitHub/GitLab}: Repositorio centralizado
    \item \textbf{Branching strategy}: GitFlow para organización de código
\end{itemize}

\subsubsection{Desarrollo y Testing}
\begin{itemize}
    \item \textbf{IDE}: Visual Studio Code, IntelliJ IDEA
    \item \textbf{Testing}: Jest, JUnit, Selenium
    \item \textbf{CI/CD}: Jenkins, GitHub Actions
\end{itemize}

\subsection{Aseguramiento de Calidad}

\subsubsection{Estándares de Código}
\begin{itemize}
    \item Revisiones de código (Code Reviews)
    \item Estándares de nomenclatura y documentación
    \item Análisis estático de código
    \item Métricas de cobertura de pruebas
\end{itemize}

\subsubsection{Proceso de Testing}
\begin{itemize}
    \item \textbf{Test-Driven Development (TDD)}: Desarrollo guiado por pruebas
    \item \textbf{Pruebas automatizadas}: Unitarias, integración y funcionales
    \item \textbf{Pruebas manuales}: Exploratorias y de usabilidad
    \item \textbf{Performance testing}: Pruebas de carga y rendimiento
\end{itemize}

\subsection{Comunicación y Seguimiento}

\subsubsection{Reuniones Regulares}
\begin{itemize}
    \item \textbf{Daily Standup}: Reuniones diarias de 15 minutos
    \item \textbf{Sprint Planning}: Planificación al inicio de cada sprint
    \item \textbf{Sprint Review}: Demostración al final de cada sprint
    \item \textbf{Sprint Retrospective}: Análisis de mejoras
\end{itemize}

\subsubsection{Reportes y Métricas}
\begin{itemize}
    \item Burndown charts para seguimiento del progreso
    \item Reportes semanales de avance
    \item Métricas de calidad y rendimiento
    \item Documentación de lecciones aprendidas
\end{itemize}

\subsection{Gestión de Riesgos}

\begin{itemize}
    \item Identificación temprana de riesgos técnicos y de proyecto
    \item Plan de contingencia para escenarios críticos
    \item Backup y recuperación de datos
    \item Documentación de decisiones técnicas importantes
\end{itemize}