% EJEMPLO DE USO DEL GLOSARIO
% ===========================
% Este archivo muestra cómo referenciar términos del glosario en el texto

\section{Ejemplo de Uso del Glosario}

\subsection{Referencias Simples}
% Para hacer referencia a un término del glosario, usa \gls{término}

En este proyecto utilizaremos \gls{api} para la comunicación entre el \gls{frontend} 
y el \gls{backend}. La \gls{database} será implementada usando \gls{postgresql}.

\subsection{Referencias Múltiples}
% Primera referencia mostrará la definición completa, referencias subsecuentes solo el término

La metodología \gls{agile} con \gls{scrum} permitirá entregas iterativas. 
En el segundo \gls{sprint}, implementaremos las \gls{api} principales.

\subsection{Referencias en Plural}
% Para plural, usa \glspl{término}

Los \glspl{stakeholder} de UNICEF evaluarán los \glspl{deliverable} en cada \gls{milestone}.

\subsection{Solo la Descripción}
% Para mostrar solo la descripción, usa \glsdesc{término}

\gls{jwt}: \glsdesc{jwt}

\subsection{Forzar Forma Completa}
% Para forzar la forma completa (término + descripción), usa \Gls{término}

\Gls{microservices} será la arquitectura elegida para este proyecto.

% COMANDOS ÚTILES DEL GLOSARIO:
% =============================
% \gls{término}        - Referencia normal (primera vez: completa, después: solo término)
% \Gls{término}        - Igual que \gls pero con primera letra en mayúscula
% \glspl{término}      - Forma plural
% \Glspl{término}      - Forma plural con primera letra en mayúscula
% \glsdesc{término}    - Solo la descripción
% \glslink{término}{texto} - Enlace personalizado al glosario
% \glsfirst{término}   - Siempre muestra la forma completa
% \glsreset{término}   - Resetea el término para que la próxima referencia sea completa