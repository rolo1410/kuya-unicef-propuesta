\section{Objetivos}

\subsection{Objetivo Principal}
Desarrollar e implementar la mejora y reestructuración del Sistema de Alerta Temprana (SAT) del Ministerio de Educación, Deporte y Cultura, mediante una plataforma web tecnológica modular, escalable y de acceso web, que permita gestionar, analizar y visualizar información sobre eventos adversos que afectan al sistema educativo, facilitando la toma de decisiones oportunas, conforme a los lineamientos técnicos establecidos.

\subsection{Objetivos Específicos}

\begin{enumerate}
    \item \textbf{Reestructuración y mejora de la plataforma:} Corregir errores, reestructurar y mejorar integralmente la Plataforma del SAT sobre su base tecnológica actual, asegurando su funcionamiento en una plataforma web accesible por internet.
    
    \item \textbf{Arquitectura escalable:} Implementar una arquitectura modular escalable que soporte la futura incorporación de aproximadamente 4 módulos adicionales y un aplicativo móvil.
    
    \item \textbf{Funcionalidades de gestión:} Integrar funcionalidades robustas como registro, gestión, y análisis de información relacionada con eventos adversos.
    
    \item \textbf{Integración de sistemas:} Integrar datos georreferenciados, automatizar reportes y garantizar la interoperabilidad con sistemas institucionales como AMIE y GIEE.
    
    \item \textbf{Implementación de módulos funcionales:} Implementar tres módulos funcionales en la primera fase: ADMINISTRACIÓN, EVENTOS ADVERSOS y VISUALIZADOR.
    
    \item \textbf{Sistema de roles:} Desarrollar la plataforma con roles diferenciados (Técnico de Riesgos Distrital, Zonal y Nacional) para una gestión descentralizada y eficiente.
    
    \item \textbf{Trazabilidad de información:} Garantizar la trazabilidad de la información de eventos, incluyendo la edición, revisión, y aprobación según el rol del usuario.
    
    \item \textbf{Capacitación:} Diseñar e implementar un plan de capacitación técnica y funcional a funcionarios y usuarios del SAT.
\end{enumerate}
