\section{Análisis de Riesgos y Estrategias de Mitigación}

\subsection{Identificación de Riesgos}

\subsubsection{Riesgos Técnicos}
\begin{itemize}
    \item \textbf{Integración de sistemas:} Dificultades en la integración con sistemas existentes de UNICEF
    \item \textbf{Escalabilidad:} Limitaciones de rendimiento bajo alta carga de usuarios
    \item \textbf{Seguridad de datos:} Vulnerabilidades en el manejo de información sensible
    \item \textbf{Compatibilidad tecnológica:} Incompatibilidades con infraestructura actual
\end{itemize}

\subsubsection{Riesgos de Proyecto}
\begin{itemize}
    \item \textbf{Retrasos en cronograma:} Desviaciones en los tiempos de entrega planificados
    \item \textbf{Cambios en requerimientos:} Modificaciones significativas en el alcance del proyecto
    \item \textbf{Disponibilidad de recursos:} Falta de personal especializado o recursos técnicos
    \item \textbf{Comunicación:} Fallas en la coordinación entre equipos
\end{itemize}

\subsubsection{Riesgos Operacionales}
\begin{itemize}
    \item \textbf{Adopción por usuarios:} Resistencia al cambio por parte de los usuarios finales
    \item \textbf{Capacitación:} Insuficiente entrenamiento del personal
    \item \textbf{Mantenimiento:} Dificultades en el soporte post-implementación
\end{itemize}

\subsection{Estrategias de Mitigación}

\begin{table}[h]
\centering
\begin{tabular}{|p{3cm}|p{2cm}|p{5cm}|p{4cm}|}
\hline
\textbf{Riesgo} & \textbf{Probabilidad} & \textbf{Estrategia de Mitigación} & \textbf{Responsable} \\
\hline
Integración de sistemas & Media & Realizar pruebas de integración tempranas y mantener comunicación constante con el equipo de IT de UNICEF & Líder Técnico \\
\hline
Retrasos en cronograma & Media & Implementar metodología ágil con sprints cortos y revisiones frecuentes & Project Manager \\
\hline
Seguridad de datos & Alta & Aplicar estándares de seguridad internacionales, auditorías regulares y cifrado de datos & Especialista en Seguridad \\
\hline
Cambios en requerimientos & Media & Establecer procesos de control de cambios y documentación detallada & Analista de Negocio \\
\hline
Adopción por usuarios & Alta & Involucrar usuarios en el diseño, realizar capacitaciones y crear material de apoyo & UX Designer / Capacitador \\
\hline
\end{tabular}
\caption{Matriz de Riesgos y Mitigación}
\end{table}

\subsection{Plan de Contingencia}

\subsubsection{Procedimientos de Escalamiento}
\begin{enumerate}
    \item \textbf{Nivel 1:} Resolución a nivel de equipo técnico
    \item \textbf{Nivel 2:} Escalamiento a líder de proyecto
    \item \textbf{Nivel 3:} Involucramiento de stakeholders de UNICEF
    \item \textbf{Nivel 4:} Activación de plan de contingencia corporativo
\end{enumerate}

\subsubsection{Comunicación de Crisis}
\begin{itemize}
    \item Notificación inmediata a stakeholders principales
    \item Reuniones de emergencia dentro de 24 horas
    \item Reportes de estado cada 48 horas hasta resolución
    \item Documentación completa de lecciones aprendidas
\end{itemize}

\subsection{Monitoreo y Control}

\subsubsection{Indicadores de Riesgo}
\begin{itemize}
    \item Desviación del cronograma mayor al 10\%
    \item Incremento del presupuesto mayor al 5\%
    \item Más de 3 defectos críticos en producción
    \item Tiempo de respuesta del sistema mayor a 3 segundos
\end{itemize}

\subsubsection{Frecuencia de Revisión}
\begin{itemize}
    \item \textbf{Revisiones semanales:} Estado de riesgos identificados
    \item \textbf{Revisiones mensuales:} Evaluación integral de riesgos
    \item \textbf{Revisiones por hitos:} Análisis completo al finalizar cada fase
\end{itemize}