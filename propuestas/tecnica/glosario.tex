% =============================================================================
% GLOSARIO DE TÉRMINOS - PROPUESTA TÉCNICA UNICEF
% =============================================================================
% Este archivo contiene todas las definiciones de términos técnicos
% utilizados en la propuesta. Se utiliza el paquete glossaries para
% manejo automático de referencias y ordenamiento.
% =============================================================================

% Configuración del glosario
\makeglossaries

% =============================================================================
% TÉRMINOS TÉCNICOS GENERALES
% =============================================================================

\newglossaryentry{api}
{
    name={API},
    description={Application Programming Interface. Conjunto de protocolos, rutinas y herramientas para construir aplicaciones de software. Define las formas en que los componentes de software deben interactuar}
}

\newglossaryentry{backend}
{
    name={Backend},
    description={Parte del sistema que maneja la lógica del servidor, bases de datos y arquitectura de la aplicación. No es visible directamente para el usuario final}
}

\newglossaryentry{frontend}
{
    name={Frontend},
    description={Parte del sistema con la que interactúa directamente el usuario. Incluye la interfaz gráfica, navegación y experiencia del usuario}
}

\newglossaryentry{framework}
{
    name={Framework},
    description={Marco de trabajo o conjunto de herramientas, bibliotecas y convenciones que facilitan el desarrollo de aplicaciones de software}
}

\newglossaryentry{database}
{
    name={Base de Datos},
    description={Sistema organizado para almacenar, gestionar y recuperar información de manera eficiente y segura}
}

% =============================================================================
% METODOLOGÍAS Y PROCESOS
% =============================================================================

\newglossaryentry{agile}
{
    name={Metodología Ágil},
    description={Enfoque de desarrollo de software que enfatiza la colaboración, la flexibilidad y la entrega incremental de funcionalidades}
}

\newglossaryentry{scrum}
{
    name={Scrum},
    description={Marco de trabajo ágil para gestionar el desarrollo de productos complejos, basado en sprints y equipos autoorganizados}
}

\newglossaryentry{devops}
{
    name={DevOps},
    description={Conjunto de prácticas que combina desarrollo de software (Dev) y operaciones de TI (Ops) para acortar el ciclo de vida del desarrollo}
}

\newglossaryentry{cicd}
{
    name={CI/CD},
    description={Continuous Integration/Continuous Deployment. Práctica de automatizar la integración y despliegue de código para mejorar la calidad y velocidad de entrega}
}

% =============================================================================
% TECNOLOGÍAS WEB Y DESARROLLO
% =============================================================================

\newglossaryentry{html}
{
    name={HTML},
    description={HyperText Markup Language. Lenguaje de marcado estándar para crear páginas web y aplicaciones web}
}

\newglossaryentry{css}
{
    name={CSS},
    description={Cascading Style Sheets. Lenguaje utilizado para describir la presentación y el diseño de documentos HTML}
}

\newglossaryentry{javascript}
{
    name={JavaScript},
    description={Lenguaje de programación interpretado que permite crear contenido dinámico e interactivo en páginas web}
}

\newglossaryentry{nodejs}
{
    name={Node.js},
    description={Entorno de ejecución para JavaScript construido sobre el motor V8 de Chrome, que permite ejecutar JavaScript en el servidor}
}

\newglossaryentry{react}
{
    name={React},
    description={Biblioteca de JavaScript desarrollada por Facebook para construir interfaces de usuario, especialmente aplicaciones web de una sola página}
}

% =============================================================================
% BASES DE DATOS Y ALMACENAMIENTO
% =============================================================================

\newglossaryentry{sql}
{
    name={SQL},
    description={Structured Query Language. Lenguaje de programación diseñado para gestionar y manipular bases de datos relacionales}
}

\newglossaryentry{nosql}
{
    name={NoSQL},
    description={Término que describe bases de datos no relacionales diseñadas para manejar grandes volúmenes de datos y estructuras flexibles}
}

\newglossaryentry{mongodb}
{
    name={MongoDB},
    description={Sistema de base de datos NoSQL orientado a documentos, que almacena datos en formato similar a JSON}
}

\newglossaryentry{postgresql}
{
    name={PostgreSQL},
    description={Sistema de gestión de bases de datos relacionales y orientado a objetos de código abierto}
}

% =============================================================================
% SEGURIDAD Y AUTENTICACIÓN
% =============================================================================

\newglossaryentry{ssl}
{
    name={SSL/TLS},
    description={Secure Sockets Layer/Transport Layer Security. Protocolos criptográficos que proporcionan comunicaciones seguras por internet}
}

\newglossaryentry{oauth}
{
    name={OAuth},
    description={Estándar abierto para autorización que permite a aplicaciones de terceros obtener acceso limitado a servicios web}
}

\newglossaryentry{jwt}
{
    name={JWT},
    description={JSON Web Token. Estándar para transmitir información de forma segura entre partes como un objeto JSON compacto y autónomo}
}

\newglossaryentry{https}
{
    name={HTTPS},
    description={HyperText Transfer Protocol Secure. Protocolo de comunicación seguro que utiliza cifrado SSL/TLS para proteger la transferencia de datos}
}

% =============================================================================
% ARQUITECTURA Y INFRAESTRUCTURA
% =============================================================================

\newglossaryentry{microservices}
{
    name={Microservicios},
    description={Arquitectura de software que estructura una aplicación como un conjunto de servicios pequeños, independientes y débilmente acoplados}
}

\newglossaryentry{docker}
{
    name={Docker},
    description={Plataforma de virtualización a nivel de sistema operativo que permite empaquetar aplicaciones y sus dependencias en contenedores}
}

\newglossaryentry{kubernetes}
{
    name={Kubernetes},
    description={Sistema de orquestación de contenedores de código abierto para automatizar el despliegue, escalado y gestión de aplicaciones}
}

\newglossaryentry{cloud}
{
    name={Cloud Computing},
    description={Entrega de servicios de computación a través de internet, incluyendo servidores, almacenamiento, bases de datos y software}
}

\newglossaryentry{aws}
{
    name={AWS},
    description={Amazon Web Services. Plataforma de servicios en la nube que ofrece potencia de cómputo, almacenamiento de bases de datos y distribución de contenido}
}

% =============================================================================
% TÉRMINOS ESPECÍFICOS DE UNICEF
% =============================================================================

\newglossaryentry{beneficiario}
{
    name={Beneficiario},
    description={Persona que recibe directamente los servicios, programas o asistencia proporcionados por UNICEF}
}

\newglossaryentry{programa}
{
    name={Programa},
    description={Conjunto coordinado de actividades diseñadas para lograr objetivos específicos de desarrollo infantil y protección de derechos}
}

\newglossaryentry{monitoreo}
{
    name={Monitoreo},
    description={Proceso sistemático de recolección y análisis de información para seguir el progreso de programas y actividades}
}

\newglossaryentry{indicador}
{
    name={Indicador},
    description={Medida cuantitativa o cualitativa que proporciona información sobre el progreso hacia el logro de objetivos específicos}
}

% =============================================================================
% GESTIÓN DE PROYECTOS
% =============================================================================

\newglossaryentry{sprint}
{
    name={Sprint},
    description={Período de tiempo fijo (generalmente 1-4 semanas) durante el cual se completa un conjunto específico de trabajo en metodologías ágiles}
}

\newglossaryentry{stakeholder}
{
    name={Stakeholder},
    description={Persona, grupo u organización que puede afectar o ser afectado por las actividades y decisiones de un proyecto}
}

\newglossaryentry{milestone}
{
    name={Hito (Milestone)},
    description={Punto significativo en el cronograma del proyecto que marca la finalización de una fase o entregable importante}
}

\newglossaryentry{deliverable}
{
    name={Entregable},
    description={Producto, servicio o resultado único y verificable que debe ser producido para completar un proceso, fase o proyecto}
}

% =============================================================================
% CALIDAD Y TESTING
% =============================================================================

\newglossaryentry{qa}
{
    name={QA},
    description={Quality Assurance. Proceso sistemático para determinar si un producto o servicio cumple con los requisitos especificados}
}

\newglossaryentry{unittest}
{
    name={Prueba Unitaria},
    description={Método de testing que verifica el funcionamiento correcto de componentes individuales del software}
}

\newglossaryentry{uat}
{
    name={UAT},
    description={User Acceptance Testing. Fase final de testing donde usuarios finales verifican que el sistema cumple con sus requisitos}
}

\newglossaryentry{regression}
{
    name={Prueba de Regresión},
    description={Tipo de testing que verifica que nuevos cambios no afecten negativamente funcionalidades existentes}
}

% =============================================================================
% INTERFAZ Y EXPERIENCIA DE USUARIO
% =============================================================================

\newglossaryentry{ui}
{
    name={UI},
    description={User Interface. Interfaz de usuario que abarca todos los elementos visuales e interactivos con los que el usuario interactúa}
}

\newglossaryentry{ux}
{
    name={UX},
    description={User Experience. Experiencia global que tiene una persona al usar un producto, sistema o servicio}
}

\newglossaryentry{responsive}
{
    name={Diseño Responsivo},
    description={Enfoque de diseño web que hace que las páginas se vean bien en todos los dispositivos y tamaños de pantalla}
}

\newglossaryentry{wireframe}
{
    name={Wireframe},
    description={Representación esquemática de la estructura y funcionalidad de una página web o aplicación, sin elementos de diseño visual}
}

% =============================================================================
% COMANDO PARA IMPRIMIR EL GLOSARIO
% =============================================================================
% Para incluir el glosario en el documento principal, usar:
% \printglossary[title={Glosario de Términos}]