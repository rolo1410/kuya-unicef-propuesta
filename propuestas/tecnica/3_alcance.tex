\section{Alcance}
\subsection{Funcionalidades principales}
El alcance principal es la mejora funcional y reestructuración del SAT a través del desarrollo e implementación de nuevos módulos y la reestructuración de los existentes. El desarrollo debe realizarse sobre la base tecnológica actual del SAT, garantizando interoperabilidad con sistemas institucionales, trazabilidad y escalabilidad.

Se espera que la solución tecnológica propuesta incluya, pero no se limite a, los siguientes componentes:

\textbf{Administración:} Permite a usuarios designados (Técnico de Riesgos Nacional) modificar parámetros de la herramienta.
\begin{itemize}
    \item Gestión de catálogos
    \item Gestión de macro eventos
    \item Gestión de usuarios
    \item Carga masiva de eventos
\end{itemize}

\textbf{Eventos Adversos:} Permite el registro detallado, gestión (modificación, seguimiento, cierre), revisión y aprobación de información sobre eventos adversos y la gestión de alojamientos temporales.
\begin{itemize}
    \item Agregar Nuevo Evento
    \item Consulta de Eventos Creados
    \item Alojamientos Temporales en IE
\end{itemize}

\textbf{Visualizador:} Permite la visualización de información consolidada en formato ejecutivo con mapas interactivos, filtros, buscadores y la generación de reportes automatizados para la toma de decisiones.
\begin{itemize}
    \item Visualización
\end{itemize}

\subsection{Gestión de usuarios}
Se deben implementar tres roles de usuarios con acceso diferenciado a la herramienta:

\textbf{Técnico de Riesgos Nacional (Administrador):} Encargado de la gestión de información a nivel nacional. Puede realizar modificaciones, revisiones y aprobaciones de cualquier evento y tiene acceso a todos los módulos (ADMINISTRACIÓN, EVENTOS ADVERSOS, VISUALIZACIÓN). Es el único que puede eliminar eventos y administrar el sistema.

\textbf{Técnico de Riesgos Zonal:} Encargado de la gestión de información a nivel zonal (dentro de su zona). Puede modificar y revisar eventos dentro de su zona, y tiene acceso al módulo EVENTOS ADVERSOS y VISUALIZACIÓN.

\textbf{Técnico de Riesgos Distrital:} Encargado de la gestión de información a nivel distrital (dentro de su distrito). Puede modificar eventos dentro de su distrito, y tiene acceso al módulo EVENTOS ADVERSOS y VISUALIZACIÓN.
